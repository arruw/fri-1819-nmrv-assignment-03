\documentclass[runningheads]{llncs}
%
\usepackage{graphicx}
\usepackage[utf8]{inputenc}
\usepackage[english]{babel}
\usepackage{hyperref}
\usepackage{listings}
\usepackage{siunitx}
\usepackage{csvsimple}
% 
\begin{document}
% 
\title{Report 3: Correlation Filter Tracking}
\author{Matjaž Mav}
\institute{Advanced Computer Vision Methods}
%
\maketitle
%
 
\section{Introduction}
In this exercise we had task to implement correlation filter tracker and then integrate with the \textit{Tracking toolkit lite} \footnote{\url{https://github.com/alanlukezic/tracking-toolkit-lite}}. Additionally we implemented MOSSE tracker and some of our own ideas. At the end we performed some analysis.

\section{Correlation Filter Tracking}
The idea behind correlation filter tracking is to find filter that ideally correlates with an target. Ideal correlation is represented as Gaussian peak in the center of the target. And after we compute initial filter, we want to localize target on the next frame and slowly update filter. Update is required so that filter can adjust to the change of the target appearance. All of the computation was performed in the frequency-domain so computation is relatively fast.

\subsection{Parameters}
Here are described key three parameters used in our tracker: 
\begin{enumerate}
    \item \textbf{\textit{sigma}} - Gaussian function parameter, that defines how steep the Gaussian peak is. With low value of this parameter ($< 1$), our tracker just bounce around, if we increase it, tracking get smoother.
    \item \textbf{\textit{alpha}} - Update speed of the correlation filter, enables tracking target that changes appearance over the time. We implemented MOSSE tracker, so in our case we updated nominator and denominator separately. Before we actually applied this parameter, we weighted it with the PSR (Peak to Side lobe Ration) \footnote{We didn't actually calculated PSR, but instead just took the value with the highest correlation. We found out that in our case, if the value felt below 5, our tracker was very likely off the target. Eater the target disappeared or tracker failed.} measure. And if the PSR was below some limit, we completely stopped updating the correlation filter. With this we protected the filter to slowly drift off the target.
    \item \textbf{\textit{scale}} - Ration between the target region (actual target bounding box) and the search region (some extended area around target region). This parameter effects tracking stability and speed. It also helps in case that the target disappears from the frame. With greater value of this parameter, tracker is more likely to localize the target when it appears again.  
\end{enumerate}

Additionally we can change (4) the height of the Gaussian peak (\textbf{\textit{peak = 100}}), (5) the lower PSR limit in percentage (\textbf{\textit{psr = 0.05}}) and the regularization parameter that prevents overfitting the correlation filter (\textbf{\textit{lambda = 1e-5}}).

\subsection{Testing different parameters}
From the papers, that are listed in the exercise instructions, we found that the optimal parameters for the MOSSE tracker was already found and so we started by using them ($sigma = 2$, $alpha = 0.125$, $scale = 2$). 

\subsection{Performance analysis}

\begin{table}
    \begin{center}
    \begin{tabular}{l c c c c c}
    \hline 
    {\bf Sequence} & {\bf Overlap} & {\bf Failures} & {\bf FPS} & {\bf F20} & {\bf Frames} \\
    \hline 
    ball & 0.40 & 0 & 604.89 & 4.69 & 602 \\
    basketball & 0.46 & 5 & 290.86 & 1.32 & 725 \\
    bicycle & 0.49 & 0 & 1327.93 & 1.36 & 271 \\
    bolt & 0.63 & 3 & 451.45 & 0.85 & 350 \\
    car & 0.40 & 0 & 1219.36 & 1.16 & 252 \\
    david & 0.69 & 1 & 202.34 & 1.30 & 770 \\
    diving & 0.32 & 2 & 279.77 & 0.75 & 219 \\
    drunk & 0.38 & 0 & 182.45 & 1.20 & 1210 \\
    fernando & 0.43 & 5 & 146.73 & 1.32 & 292 \\
    fish1 & 0.37 & 13 & 912.09 & 1.53 & 436 \\
    fish2 & 0.25 & 5 & 344.86 & 1.94 & 310 \\
    gymnastics & 0.58 & 3 & 416.52 & 0.71 & 207 \\
    hand1 & 0.41 & 7 & 535.32 & 0.89 & 244 \\
    hand2 & 0.43 & 14 & 819.67 & 1.94 & 267 \\
    jogging & 0.72 & 1 & 454.71 & 1.03 & 307 \\
    motocross & 0.49 & 1 & 168.21 & 1.03 & 164 \\
    polarbear & 0.50 & 0 & 638.04 & 0.92 & 371 \\
    skating & 0.50 & 2 & 459.16 & 0.73 & 400 \\
    sphere & 0.73 & 1 & 215.42 & 0.90 & 201 \\
    sunshade & 0.74 & 2 & 854.10 & 1.07 & 172 \\
    surfing & 0.41 & 0 & 1410.87 & 1.80 & 282 \\
    torus & 0.59 & 6 & 834.31 & 1.07 & 264 \\
    trellis & 0.65 & 3 & 549.17 & 2.18 & 569 \\
    tunnel & 0.33 & 0 & 522.36 & 1.42 & 731 \\
    woman & 0.77 & 1 & 638.37 & 1.18 & 597 \\
    \hline 
    {\bf Average} & 0.51 & 3.00 & 579.16 & 1.37 & 408.52 \\
    \hline 
    \end{tabular}
    \end{center}
    \end{table}
    

\section{Conclusion}



\end{document}